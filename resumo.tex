\begin{resumo}

{\noindent A Qualidade de Vida (QV) é definida pela Organização Mundial de Saúde (OMS) como algo subjetivo e de natureza multidimensional, visto que vários fatores a influenciam. Estudos recentes demonstram que a condição bucal pode atuar como um fator modificador para a QV de indivíduos portadores de doenças crônicas. Objetivo: Avaliar o impacto da saúde bucal sobre a qualidade de vida de pacientes com Doença Renal Crônica (DRC) em Hemodiálise (HD). Métodos: Trata-se de um estudo caso controle de caráter observacional e transversal, com aplicação de questionários estruturados para 100 pacientes com DRC em hemodiálise (Grupo DRC) pareados com 100 pacientes controles (Grupo Controle). Os dados demográficos e de escolaridade foram obtidos em formulário elaborado especificamente para a pesquisa. A classificação da condição socioeconômica seguiu os critérios da Associação Brasileira de Empresas de Pesquisas. A qualidade de vida, bem como, o impacto da saúde bucal sobre a QV foi avaliado através dos seguintes questionários respectivamente: 1) “\textit{The Short Form Health Survey} (SF-36)” e 2) “\textit{Oral Health Impact Profile-14} (OHIP-14)”. Resultados: A média total do OHIP-14 para o grupo controle foi de 6,06 ($\pm$ 7,44) e de 4,67 ($\pm$ 6,52) para o grupo DRC. Ao comparar os valores internos do OHIP-14 entre os grupos controle e DRC, não foi observado diferença estatisticamente significativa entre os grupos. Em relação a qualidade de vida geral avaliada pelo SF-36, o grupo DRC obteve os piores scores para os domínios “capacidade funcional” e “limitação por aspectos físicos”. Em relação a QV geral no grupo controle, este grupo apresentou pior pontuação no domínio de “limitação por aspectos emocionais”. Foi observada correlação negativa fraca entre as características odontológicas e os valores obtidos do questionário OHIP-14. Conclusões: Os pacientes com DRC em HD apresentaram um baixo impacto da saúde bucal sobre a qualidade de vida. Além disso, pacientes DRC têm uma pior Qualidade de Vida Relacionada a Saúde (QVRS) nos domínios internos “capacidade funcional” e “limitação por aspectos físicos”, sendo que as mulheres e idosos manifestam mais a influência da doença. No sentido de que o tratamento aos DRC não deve visar somente a sobrevivência, mas também maximizar a reabilitação e a QV, sugere-se que uma atenção integral aliada a uma equipe multidisciplinar possa contribuir para o sucesso no tratamento da doença base e uma melhoria na qualidade de vida geral destes indivíduos.\\

}
{\noindent \textbf{Palavras-chave:}  Qualidade de vida. Impacto da saúde bucal. Doença renal crônica.}

\end{resumo}