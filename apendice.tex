\appendix
%Apendice A -----------------------------------------------------------------------------------------------------
\chapter{Resultados dos Tempos de Execução}

\section{Ferramenta \textit{Fast Weka}}
A ferramenta \textit{Fast Weka} tem também como objetivo ser de fácil utilização a todos aqueles que necessitam de uma ferramenta de mineração de dados, ou que já utilizam a ferramenta Weka, mas que tem suas tarefas de mineração demandando de um alto tempo de execução e buscam melhor desempenho.

Assim, como principal e única mudança visual realizada na ferramenta Weka, ocorreu a criação de uma tela inicial de configuração, onde são definidas as configurações de execução concorrente ou paralela, conforme a figura \ref{fig:telainicial}.

\begin{figure}[h!]

\centering
\caption{Tela Inicial \textit{Fast Weka}}

\includegraphics[width=0.8\textwidth,natwidth=610,natheight=642]{wekap2p.png}

\text{\footnotesize Fonte: O autor}
\label{fig:telainicial}
\end{figure}

Está tela é carregada toda vez que o programa é iniciado com a execução do arquivo fastweka.jar, ou por meio da linha de comando:
\begin{Verbatim}[frame=single]
java -cp fastweka.jar weka.gui.GUIChooser
\end{Verbatim}

Por padrão, a opção \textit{Application Running Locally} está pré-selecionada, que é utilizada quando deseja-se uma execução concorrente na maquina em questão (modo multi-\textit{threads}), o campo \textit{Number of Concurrent Requests} inicializa com o número máximo de núcleos do computador, considerando os núcleos reais e virtuais, podendo receber um valor entre um e o número máximo de núcleos do computador, o qual vai delimitar o número de execuções concorrente (\textit{Threads}).

A segunda opção disponível é \textit{Application Running on the P2P Network}, quando deseja-se realizar uma execução de modo paralelo com redes P2P, com está opção selecionada o campo citado acima é desabilitado e são habilitados os campos: \textit{IP/Name Rendezvous}, \textit{IP/Name Relay}, os dois campos \textit{Port} e também o campo \textit{Number of Peers}.

O campo \textit{Number of Peers} é responsável por definir o número de pares da rede P2P, os quais devem estar disponíveis para a execução das tarefas de mineração de dados, o mesmo aceita valores entre 2 até N.  Os outros campos supracitados, identificam os endereços de IP/Nome e Porta, do \textit{Rendezvous} e \textit{Relay} que são responsáveis pela concentração dos anúncios e informações de roteamento, respectivamente, podendo os dois estarem definidos no mesmo endereço.

Para iniciar a aplicação em modo \textit{Rendezvous}/\textit{Relay} deve-se utilizar o fastweka.jar por meio da linha de comando:
\begin{Verbatim}[frame=single]
java -cp fastweka.jar weka.p2p.RunRendezvous
\end{Verbatim}

Com os pares da rede P2P, também deve ser usado o fastweka.jar, porém com a linha de comando:
\begin{Verbatim}[frame=single]
java -cp fastweka.jar weka.p2p.StartPeer
\end{Verbatim}

Para que os pares encontrem o endereço do \textit{Rendezvous} e \textit{Relay}, e necessário que existam os arquivos: relay.txt e seeds.txt, no mesmo diretório onde encontra-se o arquivo fastweka.jar, contendo a informação do IP/Nome e Porta (tcp://IP/Nome:Porta), onde estão localizados os mesmo, exemplo:
\begin{Verbatim}[frame=single]
tcp://localhost:9701
\end{Verbatim}