\chapter{REVISÃO DA LITERATURA}
\section{QUALIDADE DE VIDA}
\subsection{Histórico da Qualidade de Vida}
O termo Qualidade de Vida (\sigla{QV}{Qualidade de Vida}) foi inserido na literatura em meados de 1960 e se destacou nas últimas décadas. Elkington foi o responsável por uma das primeiras publicações sobre o tema em um editoral nos anais de medicina. Neste editorial, o autor abordou questões desejadas pela medicina: aumento do sucesso do tratamento clínico, ausência de morte e uma vida com qualidade. Alguns anos depois, em meados de 1980, surge o termo "Qualidade de Vida Relacionada à Saúde'' (\sigla{QVRS}{Qualidade de Vida Relacionada à Saúde}) descrito inicialmente por Torrance como subconjunto da QV. Por consequência, os termos "saúde", "saúde percebida", "estado de saúde", “QVRS” e “QV” são tratados como sinônimo por muitos pesquisadores e clínicos (POST, \citeyear{post2014definitions}).

\begin{table}[!h]
\centering
\caption{Descrição dos objetivos dos oito domínios que compõem o SF-36}
\label{my-label}
\begin{tabular}{llll}
\hline
 \makecell[l]{\textbf{Domínios}} &  \textbf{Número de Questões}&  \textbf{Objetivo do domínio}&   \\ \hline

 \makecell[l]{Capacidade funcional}&  \makecell[r]{10}&  \makecell*[{{p{5.5cm}}}]{Mensurar a limitação para executar atividades que envolvam a capacidade física.} &  \\ \hline

 \makecell[l]{Aspectos físicos}& \makecell[r]{4}& 		\makecell*[{{p{5.5cm}}}]{Mensurar a limitação em saúde devido a problemas físicos, ao tipo e à quantidade do trabalho realizado.}&  \\ \hline

 \makecell[l]{Dor}& \makecell[r]{2}& 					\makecell*[{{p{5.5cm}}}]{Mensurar a intensidade e desconforto causados pela dor.}&  \\ \hline

 \makecell[l]{Estado geral da saúde}& \makecell[r]{5}& 	\makecell*[{{p{5.5cm}}}]{Mensurar a percepção geral da saúde.}&  \\ \hline
 
 \makecell[l]{Vitalidade} & \makecell[r]{4}& 			\makecell*[{{p{5.5cm}}}]{Mensurar níveis de energia e fadiga.}& \\ \hline
 
 \makecell[l]{Aspectos sociais} & \makecell[r]{2}& \makecell*[{{p{5.5cm}}}]{Mensurar o impacto dos problemas físicos e emocionais nas atividades sociais.}& \\ \hline
 
\end{tabular}
\end{table}