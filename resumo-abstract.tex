\begin{abstract}

{\noindent The Quality of Life (QoL) is defined by the World Health Organization (WHO) as something subjective and multidimensional nature, since several factors influence it. Considering this multidimensional aspect, the QoL can also be influenced by the oral condition. Recent studies demonstrate that the oral condition may also act as a modifying factor for the QoL of individuals with chronic diseases. Objective: To evaluate the impact of oral health on a quality of life of patients with Chronic Renal Disease (CKD) on Hemodialysis (HD). Methods: It is a control-case study of observational and transversal character, with the application of structured question naires for 100 patients with CKD in hemodialysis (CKD group) paired with 100 control patients (Control Group). The demographic and educational data are obtained in a form prepared specifically for research. The classification of the socioeconomic condition followed the criteria of the Brazilian Association of Research Companies. The quality of life as well as the impact of oral health on QoL were evaluated through the following questionnaires, respectively: 1) The Short Form Health Survey (SF-36) and 2) Oral Health Impact Profile-14 (OHIP -14). 
Results: The total mean of OHIP-14 for the control group was 6.06 ($\pm$ 7.44) and 4.67 ($\pm$ 6.52) for the CKD group. Regarding the general quality of life evaluated by the SF-36, the  CKD group had the worst scores for the domains “functional capacity” and “limitation by physical aspects”. In relation to the general QOL in the control group, this group presented a worse score in the domain of “limitation by emotional aspects”. A negative correlation was observed between dental characteristics and the values obtained from the OHIP-14 questionnaire. Conclusions: Patients with CKD in HD have no impact on oral health on quality of life. In addition, CKD patients have a poorer Health-Related Quality of Life (HRQoL) in the internal domains “functional capacity” and “limitation by physical aspects”, with women and the elderly manifesting more of the influence of the disease. In the sense that treatment for CKD should not only focus on survival, but also on maximizing rehabilitation and QoL, it is suggested that comprehensive care combined with a multidisciplinary team can contribute to success in treating underlying disease and an improvement in quality of life of these individuals.
\\

}

{\noindent \textbf{Keywords:}  Quality of life. Impact of oral health. Chronic kidney disease.}
\end{abstract}