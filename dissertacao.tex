%% abtex2-modelo-trabalho-academico.tex, v-1.9.2 laurocesar
%% Copyright 2012-2014 by abnTeX2 group at http://abntex2.googlecode.com/ 
%%
%% This work may be distributed and/or modified under the
%% conditions of the LaTeX Project Public License, either version 1.3
%% of this license or (at your option) any later version.
%% The latest version of this license is in
%%   http://www.latex-project.org/lppl.txt
%% and version 1.3 or later is part of all distributions of LaTeX
%% version 2005/12/01 or later.
%%
%% This work has the LPPL maintenance status `maintained'.
%% 
%% The Current Maintainer of this work is the abnTeX2 team, led
%% by Lauro César Araujo. Further information are available on 
%% http://abntex2.googlecode.com/
%%
%% This work consists of the files abntex2-modelo-trabalho-academico.tex,
%% abntex2-modelo-include-comandos and abntex2-modelo-references.bib
%%

% ------------------------------------------------------------------------
% ------------------------------------------------------------------------
% abnTeX2: Modelo de Trabalho Academico (tese de doutorado, dissertacao de
% mestrado e trabalhos monograficos em geral) em conformidade com 
% ABNT NBR 14724:2011: Informacao e documentacao - Trabalhos academicos -
% Apresentacao
% ------------------------------------------------------------------------
% ------------------------------------------------------------------------

\documentclass[
	% -- opções da classe memoir --
	12pt,				% tamanho da fonte
	openright,			% capítulos começam em pág ímpar (insere página vazia caso preciso)
	oneside,			% para impressão em verso e anverso. Oposto a oneside
	a4paper,			% tamanho do papel. 
	% -- opções da classe abntex2 --
	%chapter=TITLE,		% títulos de capítulos convertidos em letras maiúsculas
	%section=TITLE,		% títulos de seções convertidos em letras maiúsculas
	%subsection=TITLE,	% títulos de subseções convertidos em letras maiúsculas
	%subsubsection=TITLE,% títulos de subsubseções convertidos em letras maiúsculas
	% -- opções do pacote babel --
	english,			% idioma adicional para hifenização
	french,				% idioma adicional para hifenização
	spanish,			% idioma adicional para hifenização
	brazil				% o último idioma é o principal do documento
	]{abntex2}

% ---
% Pacotes básicos 
% ---
\usepackage{times}			% Usa a fonte Times new Romam			
\usepackage[T1]{fontenc}		% Selecao de codigos de fonte.
\usepackage[utf8]{inputenc}		% Codificacao do documento (conversão automática dos acentos)
\usepackage{lastpage}			% Usado pela Ficha catalográfica
\usepackage{indentfirst}		% Indenta o primeiro parágrafo de cada seção.
\usepackage{color}				% Controle das cores
\usepackage{graphicx}			% Inclusão de gráficos
\usepackage{microtype} 			% para melhorias de justificação
\usepackage{lipsum}				% para geração de dummy text
\usepackage[alf]{abntex2cite}	% Citações padrão ABNT
\usepackage{scalefnt}
\usepackage[portuguese, ruled, linesnumbered]{algorithm2e}

% ---
% Informações de dados para CAPA e FOLHA DE ROSTO
% ---
\titulo{Modelo canônico abnt latex UEPG}
\autor{Nome do autor}
\local{\textbf{PONTA GROSSA}}
\data{\textbf{2018}}
\orientador{Seu orientador}
\coorientador{Seu coorientador}
\instituicao{%
  UNIVERSIDADE ESTADUAL DE PONTA GROSSA
  \par
  SETOR DE CIÊNCIAS AGRÁRIAS E DE TECNOLOGIA
  \par
  PROGRAMA DE PÓS-GRADUAÇÃO EM COMPUTAÇÃO APLICADA}
\tipotrabalho{Dissertação (Mestrado)}
% O preambulo deve conter o tipo do trabalho, o objetivo, 
% o nome da instituição e a área de concentração 
\preambulo{Trabalho de Conclusão de Curso apresentado para a obtenção do título de Mestre em Computação Aplicada pela Universidade Estadual de Ponta Grossa - UEPG}
% ---


% ---
% Configurações de aparência do PDF final

% alterando o aspecto da cor azul
\definecolor{blue}{RGB}{41,5,195}

% informações do PDF
\makeatletter
\hypersetup{
     	%pagebackref=true,
		pdftitle={\@title}, 
		pdfauthor={\@author},
    	pdfsubject={\imprimirpreambulo},
	    pdfcreator={LaTeX with abnTeX2},
		pdfkeywords={abnt}{latex}{abntex}{abntex2}{trabalho acadêmico}, 
		colorlinks=true,       		% false: boxed links; true: colored links
    	linkcolor=blue,          	% color of internal links
    	citecolor=blue,        		% color of links to bibliography
    	filecolor=magenta,      		% color of file links
		urlcolor=blue,
		bookmarksdepth=4
}
\makeatother
% --- 

% --- 
% Espaçamentos entre linhas e parágrafos 
% --- 

% O tamanho do parágrafo é dado por:
\setlength{\parindent}{1.3cm}

% Controle do espaçamento entre um parágrafo e outro:
\setlength{\parskip}{0.2cm}  % tente também \onelineskip

% ---
% compila o indice
% ---
\makeindex
% ---

% ---
%\makeglossaries
% ---

% ----
% Início do documento
% ----
\begin{document}

% Retira espaço extra obsoleto entre as frases.
\frenchspacing 
\imprimircapa
\imprimirfolhaderosto*

\include{fichacatalografica}
\include{errata}
\include{folhadeaprovacao}
\

\vfill

\begin{flushright}
\hfill \textit{Dedico este trabalho aos meus queridos pais,\\ que por amor fizeram de tudo por mim. }
\end{flushright}

\vspace*{1cm}

\clearpage
\chapter*{Agradecimentos}
À Deus, que esteve presente em todos os momentos, me guiou com sua luz divina, ouviu minhas preces e me fortaleceu. A Ti, meu Deus, toda honra e toda glória eternamente.
\include{epigrafe}
\begin{resumo}

{\noindent A Qualidade de Vida (QV) é definida pela Organização Mundial de Saúde (OMS) como algo subjetivo e de natureza multidimensional, visto que vários fatores a influenciam. Estudos recentes demonstram que a condição bucal pode atuar como um fator modificador para a QV de indivíduos portadores de doenças crônicas. Objetivo: Avaliar o impacto da saúde bucal sobre a qualidade de vida de pacientes com Doença Renal Crônica (DRC) em Hemodiálise (HD). Métodos: Trata-se de um estudo caso controle de caráter observacional e transversal, com aplicação de questionários estruturados para 100 pacientes com DRC em hemodiálise (Grupo DRC) pareados com 100 pacientes controles (Grupo Controle). Os dados demográficos e de escolaridade foram obtidos em formulário elaborado especificamente para a pesquisa. A classificação da condição socioeconômica seguiu os critérios da Associação Brasileira de Empresas de Pesquisas. A qualidade de vida, bem como, o impacto da saúde bucal sobre a QV foi avaliado através dos seguintes questionários respectivamente: 1) “\textit{The Short Form Health Survey} (SF-36)” e 2) “\textit{Oral Health Impact Profile-14} (OHIP-14)”. Resultados: A média total do OHIP-14 para o grupo controle foi de 6,06 ($\pm$ 7,44) e de 4,67 ($\pm$ 6,52) para o grupo DRC. Ao comparar os valores internos do OHIP-14 entre os grupos controle e DRC, não foi observado diferença estatisticamente significativa entre os grupos. Em relação a qualidade de vida geral avaliada pelo SF-36, o grupo DRC obteve os piores scores para os domínios “capacidade funcional” e “limitação por aspectos físicos”. Em relação a QV geral no grupo controle, este grupo apresentou pior pontuação no domínio de “limitação por aspectos emocionais”. Foi observada correlação negativa fraca entre as características odontológicas e os valores obtidos do questionário OHIP-14. Conclusões: Os pacientes com DRC em HD apresentaram um baixo impacto da saúde bucal sobre a qualidade de vida. Além disso, pacientes DRC têm uma pior Qualidade de Vida Relacionada a Saúde (QVRS) nos domínios internos “capacidade funcional” e “limitação por aspectos físicos”, sendo que as mulheres e idosos manifestam mais a influência da doença. No sentido de que o tratamento aos DRC não deve visar somente a sobrevivência, mas também maximizar a reabilitação e a QV, sugere-se que uma atenção integral aliada a uma equipe multidisciplinar possa contribuir para o sucesso no tratamento da doença base e uma melhoria na qualidade de vida geral destes indivíduos.\\

}
{\noindent \textbf{Palavras-chave:}  Qualidade de vida. Impacto da saúde bucal. Doença renal crônica.}

\end{resumo}
\include{listas}

% ---
% inserir o sumario
% ---
\pdfbookmark[0]{\contentsname}{toc}
\tableofcontents*
\cleardoublepage
% ---

% ----------------------------------------------------------
% ELEMENTOS TEXTUAIS
% ----------------------------------------------------------
\textual

\chapter{INTRODUÇÃO}
O termo qualidade de vida é definido pela Organização Mundial de Saúde (OMS) como algo subjetivo e de natureza multidimensional, visto que vários fatores a influenciam tais como a percepção individual sobre aspectos sociais, psicológicos e físicos, incluindo a condição de saúde geral do indivíduo (The WHOQOL Group, \citeyear{whoqol1995world}). 
\chapter{OBJETIVOS}
\section{OBJETIVO GERAL}
Avaliar o impacto ...

\section{OBJETIVOS ESPECÍFICOS}
\begin{itemize}
\item Caracterizar o perfil ...; 

\item Relacionar os dados ...;

\item Avaliar a qualidade ....;

\item Avaliar o impacto da ....;
\end{itemize}
\chapter{REVISÃO DA LITERATURA}
\section{QUALIDADE DE VIDA}
\subsection{Histórico da Qualidade de Vida}
O termo Qualidade de Vida (\sigla{QV}{Qualidade de Vida}) foi inserido na literatura em meados de 1960 e se destacou nas últimas décadas. Elkington foi o responsável por uma das primeiras publicações sobre o tema em um editoral nos anais de medicina. Neste editorial, o autor abordou questões desejadas pela medicina: aumento do sucesso do tratamento clínico, ausência de morte e uma vida com qualidade. Alguns anos depois, em meados de 1980, surge o termo "Qualidade de Vida Relacionada à Saúde'' (\sigla{QVRS}{Qualidade de Vida Relacionada à Saúde}) descrito inicialmente por Torrance como subconjunto da QV. Por consequência, os termos "saúde", "saúde percebida", "estado de saúde", “QVRS” e “QV” são tratados como sinônimo por muitos pesquisadores e clínicos (POST, \citeyear{post2014definitions}).

\begin{table}[!h]
\centering
\caption{Descrição dos objetivos dos oito domínios que compõem o SF-36}
\label{my-label}
\begin{tabular}{llll}
\hline
 \makecell[l]{\textbf{Domínios}} &  \textbf{Número de Questões}&  \textbf{Objetivo do domínio}&   \\ \hline

 \makecell[l]{Capacidade funcional}&  \makecell[r]{10}&  \makecell*[{{p{5.5cm}}}]{Mensurar a limitação para executar atividades que envolvam a capacidade física.} &  \\ \hline

 \makecell[l]{Aspectos físicos}& \makecell[r]{4}& 		\makecell*[{{p{5.5cm}}}]{Mensurar a limitação em saúde devido a problemas físicos, ao tipo e à quantidade do trabalho realizado.}&  \\ \hline

 \makecell[l]{Dor}& \makecell[r]{2}& 					\makecell*[{{p{5.5cm}}}]{Mensurar a intensidade e desconforto causados pela dor.}&  \\ \hline

 \makecell[l]{Estado geral da saúde}& \makecell[r]{5}& 	\makecell*[{{p{5.5cm}}}]{Mensurar a percepção geral da saúde.}&  \\ \hline
 
 \makecell[l]{Vitalidade} & \makecell[r]{4}& 			\makecell*[{{p{5.5cm}}}]{Mensurar níveis de energia e fadiga.}& \\ \hline
 
 \makecell[l]{Aspectos sociais} & \makecell[r]{2}& \makecell*[{{p{5.5cm}}}]{Mensurar o impacto dos problemas físicos e emocionais nas atividades sociais.}& \\ \hline
 
\end{tabular}
\end{table}
\chapter{MATERIAIS E MÉTODOS}

\section{CARACTERÍSTICAS GERAIS}
texto
\include{resultado}
\include{discurssao}
\include{conclusao}
\bibliography{abntex2-modelo-references}
%---------------------
%Glossário
%---------------------
%\include{glossario}
%\setglossarystyle{tree}
%\cleardoublepage
%\phantomsection
%\addcontentsline{toc}{chapter}{\glossaryname}
%\printglossaries
%--------------------

\include{apendice}
\chapter*{\textbf{ANEXOS}}

\centering
ANEXO I
\begin{figure}[h!]
\centering
\includegraphics[width=0.62\textwidth]{imagens/anexo1}
\captionsetup{labelformat=empty}
\end{figure}
\include{indice_remissivo}

\end{document}